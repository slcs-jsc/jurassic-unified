The Juelich Rapid Spectral Simulation Code (JURASSIC) is a fast radiative transfer model for the mid-infrared spectral region \citep{Hoffmann2006}. It was used in several studies for the infrared limb sounder Michelson Interferometer for Passive Atmospheric Sounding (MIPAS) \citep{Hoffmann2005, Hoffmann2008}, Cryogenic Infrared Spectrometers and Telescopes for the Atmosphere - New Frontiers (CRISTA-NF) \citep{Hoffmann2009, Weigel2010}, and Gimballed Limb Observer for Radiance Imaging of the Atmosphere (GLORIA) \citep{Ungermann2010a} and the nadir instrument Atmospheric Infrared Sounder (AIRS) \citep{Hoffmann2009b, Grimsdell2010,Hoffmann2013}.

For fast simulations, it applies pre-calculated look-up tables of spectral emissivities and approximations to radiative transfer calculations, namely the emissivity growth approximation (EGA) \citep{Weinreb1973,Gordley1981,Marshall1994}.% and the Curtis-Godson approximation (CGA) \citep{Curtis1952,Godson1953}. 
The look-up-tables were calculated with the Reference Forward Model (RFM) \citep{Dudhia2002,Dudhia2014,Dudhia2017}, which is an exact line-by-line model specifically developed for MIPAS. For selected spectral windows, JURASSIC has been compared to the line-by-line models RFM and Karlsruhe Optimized and Precise Radiative transfer Algorithm (KOPRA) \citep{Stiller2000,Stiller2002,Hoepfner2005} and shows good agreement \citep{Griessbach2013}.

JURASSIC contains a scattering module that allows for radiative transfer simulations including single scattering on aerosol and cloud particles \citep{Griessbach2012,Griessbach2013}. Forward simulations with scattering on volcanic ash, ice and sulfate aerosol have been used to develop and characterize a volcanic ash detection method for MIPAS \citep{Griessbach2012a,Griessbach2014} and to discriminate between ice, ash, and sulfate aerosol \citep{Griessbach2016,Griessbach2018}.

Retrieval of large satellite data sets (e.g. AIRS) require plenty of computing time that can be provided by supercomputers. JURASSIC-scatter showed best performance using a hybrid MPI/openMP parallelization on the past JUROPA and JUQUEEN and the recent JURECA and JUWELS at the Jülich Supercomputing Centre (JSC), Forschungszentrum Jülich GmbH. It can also be used on workstations or laptops using either pure MPI or openMP parallelization.

This documentation is neither ready nor perfect, but it is a start to introduce you to JURASSIC(-scatter) and to enable you to work and to do science with this code package.